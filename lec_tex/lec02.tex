\section*{Lecture 2: Introduction Day 2}
\subsection*{Summary}
\paragraph{A family of signals that are the building blocks of complex signal.}
Important functions and the relationships between those simple special signals. \\Review of special signal: \textit{Delta function, step function, sinusoid, exponential, sampling property, Euler's formula, discrete time convolution.}
\vspace{5pt}
\hrule
\subsection*{Last Lecture}
\subparagraph*{Digital Time Signal} To process ANY signals using any toolkit being discussed in this course, the signals need to be in the digital format (in other words, must be sampled and not be continuous).

\subsection*{Important Functions}
\[
  \delta[n] =
  \begin{cases}
    0, & n\neq 0 \\
    1, & n=0
  \end{cases}
\]
\subparagraph*{Unit Impulse Function} In continuous time, unit impulse is only defined at t=0, with weight being \textbf{unbounded}. In discrete time, unit impulse is only defined at n=0, with weight being of \textbf{1}. 
\[
  u[n] = 
  \begin{cases}
      1, & n\geq 0\\
      0, & n< 0
  \end{cases}
\]
\subparagraph*{Unit Step Function} In continuous time, unit step is a function with weight being 1 for t $\geq$ 0 and 0 otherwise. The relationships between $u(t)$ and $\delta(t)$ is:
\begin{align}
    u(t) &= \int_{t}^{} \delta(t) \,dt  \\
    \delta(t) &= \frac{d}{dt}u(t)
\end{align}
This is similar to the discrete time, except in discrete time, you cannot use intergration or differentiation, so this relationships can be shown as:
\begin{align*}
    u[n] &= \delta[n]+\delta[n-1]+\cdots \\
         &= \sum_{k=0}^{\infty} \delta[n-k] \\
    \delta[n] &= u[n]-u[n-1]      
\end{align*}
\subsection*{Sampling Property}
\subparagraph*{Can you answer this: $x[n]\delta[n-n_0]$?} If you draw the two plots and multiply them together, you notice that it will evaluate to:
\[
    x[n]\delta[n-n_0] = 
    \begin{cases}
        x[n_0], &n=n_0 \\
        0, &else
    \end{cases}
\]
To put more things into perspective, when you have a spectrum of x[n] which has different height for different n. You can imagine that as a summation of different time shifted version of impulse with a weight being multiplied to the impulse. To put this in terms of equation:
\begin{align*}
    x[n] &= \sum_{-\infty}^{\infty}x[k]\delta[n-k] \\
         &= x[n] \ast \delta[n]
\end{align*}

\subsection*{Complex Number}
\begin{align*}
    z &= x+jy, \quad(\text{Rectangular}) \\
    z &= re^{j\theta}, \quad\quad(\text{Polar}) \\
    r &= \left\lvert z \right\rvert = \sqrt{x^2+y^2}\\
    \theta &= \angle  z = \arctan(\frac{y}{x})
\end{align*}

\subsection*{Euler's Formula}
\begin{align*}
    e^{j\theta} &= cos(\theta) + jsin(\theta) \\
    cos(\theta) &= \frac{1}{2}(e^{j\theta}+e^{-j\theta})\\
    sin(\theta) &= \frac{1}{2j}(e^{j\theta}-e^{-j\theta})
\end{align*}

\subsection*{Sinusoid}
The general function of a sinusoid is a function of sin or cosine.
\[
    x(t)=Asin(w_0 t + \phi)
\]
This function above is periodic with $T = \frac{w_0}{2\pi}=\frac{1}{f}$.

\subsection*{Exponential}
The general function of an exponential is:
\[x(t)=\mathbf{C}e^{\alpha t}\]
If $\mathbf{C, \alpha}\subset \mathbb{R}:$
\[\alpha > 0 \rightarrow x(t)\text{ is an exponential grow function.}\]
\[\alpha < 0 \rightarrow x(t)\text{ is an exponential decay function.}\]
If $\mathbf{C, \alpha}\subset \mathbb{C}:$
\begin{align*}
    \mathbf{C} &= re^{j\theta}, \mathbf{\alpha} = b+jy \\
    x(t) &= \mathbf{C}e^{\alpha t} = re^{j\theta}e^{(b+jy) t} = re^{bt}e^{j(\theta+yt)} \\
         &= re^{bt} [cos(\theta+yt)+jsin(\theta+yt)]
\end{align*}
\subsubsection*{Real Envelope} The real and the non-sinusoidal constant part of x(t), $re^{bt}$, is called the real envelope in that it is an imaginary curve that set the boundary within which the signal is contained. \\
The detail discussed above is for continuous time, for the discrete time, the only difference here is this:
\begin{align*}
    \alpha &= e^{\beta} \subset \mathbb{R}, \mathbf{C}=\phi + jw_0\subset \mathbb{C}\\
    x[n] &= \mathbf{C}e^{\beta n}, n \subset \mathbb{Z} \\
         &= \mathbf{C}\alpha^n \\
         &= |\mathbf{C}||\alpha|^n [cos(w_0n+\phi)+jsin(w_0n+\phi)] \\
    e^{jw_0n} &\equiv e^{j(w_0 + 2\pi)n}, e^{j(w_0+2\pi)n} \leq 1 = e^{2\pi n}
\end{align*}
In other words, in discrete time, there's no infinitely high frequency. Only a $2\pi$-wide range of frequency. 

\subsection*{Convolution (LTI)}
You choose 1 function to time-reverse. Start n at negative inifnity and slide it to the right to positive infinity. It's better to watch youtube video to explain on this to save typesetting everything